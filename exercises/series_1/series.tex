\documentclass[a4paper]{scrreprt}

% Uncomment to optimize for double-sided printing.
% \KOMAoptions{twoside}

% Set binding correction manually, if known.
% \KOMAoptions{BCOR=2cm}

% Localization options
\usepackage[english]{babel}
\usepackage[T1]{fontenc}
\usepackage[utf8]{inputenc}

% Enhanced verbatim sections. We're mainly interested in
% \verbatiminput though.
\usepackage{verbatim}

% PDF-compatible landscape mode.
% Makes PDF viewers show the page rotated by 90°.
\usepackage{pdflscape}

% Advanced tables
\usepackage{tabu}
\usepackage{longtable}

% Fancy tablerules
\usepackage{booktabs}

% Graphics
\usepackage{graphicx}

% Current time
\usepackage[useregional=numeric]{datetime2}

% Float barriers.
% Automatically add a FloatBarrier to each \section
\usepackage[section]{placeins}

% Custom header and footer
\usepackage{fancyhdr}
\setlength{\headheight}{15.2pt}
\pagestyle{fancyplain}

\usepackage{geometry}
\usepackage{layout}

% Math tools
\usepackage{mathtools}
% Math symbols
\usepackage{amsmath,amsfonts,amssymb}

% \fancyhf{}
% \lhead{}
% \lfoot{}
% \rfoot{}
% 
% Source code & highlighting
\usepackage{listings}

% Convenience commands
\newcommand{\mailsubject}{451670- Algorithmen, Wahrscheinlichkeit, Informationen - Practical exercise 1}
\newcommand{\maillink}[1]{\href{mailto:#1?subject=\mailsubject}
                               {#1}}

% Should use this command wherever the print date is mentioned.
\newcommand{\printdate}{\today}

\subject{451670 - Algorithmen, Wahrscheinlichkeit, Informationen}
\title{Practical exercise 1}

\author{Michael Senn \maillink{michael.senn@students.unibe.ch}}

\date{\printdate}

% Needs to be the last command in the preamble, for one reason or
% another. 
\usepackage{hyperref}


\begin{document}
\maketitle

\chapter{Series 1}

\section{Binary symmetrical channel}

Let $X$ be the number of incorrectly transmitted bits. As each transmission is
independent from the others, x follows a binomial distribution:

\[
	X \sim Bi(k, p)
\]

As such, its CDF is given by:

\[
	P(X \leq x) = \sum_{i=0}^{x}{\binom{k}{i} p^i * (1-p)^i}
\]

For uneven $k$, a correct transmission is one where less than half -
$\frac{k-1}{2}$ - bits were transmitted wrongly. Hence, the probability of the
transmission being correct is:

\[
	P(X \leq \frac{k-1}{2}) = \sum_{i=0}^{\frac{k-1}{2}}{\binom{k}{i} p^i * (1-p)^i}
\]

For both the case of $p = 0$ - ie no bit is transmitted wrongly - and the case
of $p = \frac{1}{2}$, we get the results we expect.

With $p = \frac{1}{2}$:

\begin{align*}
	P(X \leq \frac{k-1}{2}) & = \sum_{i=0}^{\frac{k-1}{2}}{\binom{k}{i} \frac{1}{2}^i * (1-\frac{1}{2})^i} \\
				& = \frac{1}{2^k} * \sum_{i=0}^{\frac{k-1}{2}}{\binom{k}{i}} \\
				& = \frac{1}{2^k} * 2^{k-1} \\
				& = \frac{1}{2}
\end{align*}

With $p = 0$:

\begin{align*}
	P(X \leq \frac{k-1}{2}) & = \sum_{i=0}^{\frac{k-1}{2}}{\binom{k}{i} 0^i * 1^{k-i}} \\
			        & = \binom{k}{0} * 0^0 * 1^k + \sum_{i=1}^{\frac{k-1}{2}}{\binom{k}{i} * 0^i * 1^{k-1}} \\
		                & = 1 + 0 \\
	                        & = 1
\end{align*}

\section{Pairwise independent $\neq$ mutually independent}

Let $A, B$ be distributed according to the discrete uniform distribution on
${0, 1}$. Let $C := A \oplus B$. This nets the following value table.
\\

\begin{tabular}{|c|c|c|}
	\hline
	A & B & C \\
	\hline
	0 & 0 & 0 \\
	\hline
	0 & 1 & 1 \\
	\hline
	1 & 0 & 1 \\
	\hline
	1 & 1 & 1 \\
	\hline
\end{tabular}

Hence it follows that:
\begin{align*}
	P(A=0) = P(B=0) = P(C=0) & = \frac{1}{2} \\
	P(A=0, B=0) = P(A=0) * P(B=0) & = \frac{1}{4} \\
	P(A=0, C=0) = P(A=0) * P(C=0) & = \frac{1}{4} \\
	P(B=0, C=0) = P(B=0) * P(C=0) & = \frac{1}{4}
\end{align*}

Meaning that $A$, $B$ and $C$ are pairwise independent. 

However:
\[
	P(A=0, B=0, C=0) = \frac{1}{4} \neq \frac{1}{8} = P(A=0) * P(B=0) * P(C=0)
\]

As such $A$, $B$ and $C$ are not mutually independent.

This also matches with what one expects intuitively, as knowing one parameter
of an XOR operation says nothing about the other two, whereas knowing any two
diretly identifies the third.

\section{Sum of dice}

Observe first that, for any number $n \in \mathbb{N}$, one can get a multiple
of six by adding a second number $m \in \mathbb{N}$, with $1 \leq m \leq 6$.

As all dice throws are independent, we assume $x$ to be the sum after the first
five dice throws. Due to the property mentioned above, there exists exactly one
number $y$ in $1 \leq y \leq 6$ such that $x + y$ is a multiple of six.

As the dice produces values between $1$ and $6$ evenly, this means there is a
chance of $\frac{1}{6}$ that the sum of six dice throws - or the sum of any
number of dice throws even - is a multiple of six.

\section{Coin tricks}

Let $C_1$, $C_2$, $C_3$ be the three coins in questions. Then we have:

\begin{align*}
	P(T | C_1) & = \frac{1}{2} \\
	P(T | C_2) & = 1 \\
	P(T | C_3) & = \frac{3}{4}
\end{align*}

Using Bayes' theorem we can then calculate the probability of $C_2$ having been
tossed, given that the toss was tails:

\begin{align*}
	P(C_2 | T) & = \frac{P(T | C_2) * P(C_2)}{\sum_{i=1}^3{P(T | C_i) * P(C_i)}} \\
		   & = \frac{1 * \frac{1}{3}}{\frac{1}{2} * \frac{1}{3} + 1 * \frac{1}{3} + \frac{3}{4} * \frac{1}{3}} \\
		   & = \frac{12}{27}
\end{align*}



\section{Simulating fair coin toss using biased coin toss}

\subsection{Algorithm and correctness}

Let $H := \texttt{Coin shows head}$, let $T := \texttt{Coin shows tails}$.

Observe the following algorithm:
\begin{itemize}
	\item Let $A$ be result of first coin toss
	\item Let $B$ be result of second coin toss
	\item Repeat both tosses in pairs, until $A \neq B$
	\item If $A = H \land B = T$: Return $H$
	\item If $A = T \land B = H$: Return $T$
\end{itemize}

This algorithm is guaranteed to produce either heads or tails with probability
$p = \frac{1}{2}$. This can be proven as follows.

Let $p$ be probability of the weighted coin producing head. Let $X$ be the
result of the first, $Y$ the result of the second coin toss.

Clearly we have:
\[
	P(X=H) = P(Y=H) = p
\]

As well as:
\[
	P(X=T) = P(Y=T) = 1-p
\]

As each toss is independent, we get:
\begin{align*}
	P(X=H, Y=T) & = P(X=H) * P(Y=T) \\
	& = p * (1-p) \\
	& = (1-p) * p \\
	& = P(X=T) * P(Y=H) \\
	& = P(X=T, Y=H)
\end{align*}

As such - no matter the probability $p$ - the chance of getting first head then
tails is equal to getting first tails then head.

\subsection{Number of usable coin tosses}

Let $k$ be the number of weighted coin tosses. Let $l := \frac{k}{2}$ the
number of formed pairs. Any such pair is usable to simulate a fair coin toss if
its two coins are not equal.

As such, the number of usable coin tosses follows the binomial distribution
with:
\[
	X \sim Bi(l, 2 * p * (1-p))
\]

The algorithm can be improved by looking at consecutive tuples of $2^n, n \in
\mathbb{N}$ coin tosses.

Let $X$ be the first pair of throws, $Y$ the second. If $X$ and $Y$ can not be
used to simulate a fair coin toss - that is $(X = HH \lor X = TT) \land (Y = HH
\lor Y = TT)$, then we can use them to simulate a fair coin toss as long as $X
\neq Y$. This is, as before, given by the independence of coin tosses:
\[
	P(X=HH, Y=TT) = p^2 * (1-p)^2 = (1-p)^2 * p^2 = P(X=TT, Y=HH)
\]

Similarly, one can expand the algorithm to any power of two.

\subsection{Maximum number of fair coin tosses}

At most one can generate $\frac{k}{2}$ fair coin tosses, if each pair is
composed of non-equal coin tosses.

\end{document}

