\documentclass[a4paper]{scrreprt}

% Uncomment to optimize for double-sided printing.
% \KOMAoptions{twoside}

% Set binding correction manually, if known.
% \KOMAoptions{BCOR=2cm}

% Localization options
\usepackage[english]{babel}
\usepackage[T1]{fontenc}
\usepackage[utf8]{inputenc}

% Enhanced verbatim sections. We're mainly interested in
% \verbatiminput though.
\usepackage{verbatim}

% PDF-compatible landscape mode.
% Makes PDF viewers show the page rotated by 90°.
\usepackage{pdflscape}

% Advanced tables
\usepackage{tabu}
\usepackage{longtable}

% Fancy tablerules
\usepackage{booktabs}

% Graphics
\usepackage{graphicx}

% Current time
\usepackage[useregional=numeric]{datetime2}

% Float barriers.
% Automatically add a FloatBarrier to each \section
\usepackage[section]{placeins}

% Custom header and footer
\usepackage{fancyhdr}

\usepackage{geometry}
\usepackage{layout}

% Math tools
\usepackage{mathtools}
% Math symbols
\usepackage{amsmath,amsfonts,amssymb}
\usepackage{amsthm}

\DeclarePairedDelimiter\abs{\lvert}{\rvert}

\pagestyle{plain}
% \fancyhf{}
% \lhead{}
% \lfoot{}
% \rfoot{}
% 
% Source code & highlighting
\usepackage{listings}

% Convenience commands
\newcommand{\mailsubject}{451670- Algorithmen, Wahrscheinlichkeit, Informationen - Series 4}
\newcommand{\maillink}[1]{\href{mailto:#1?subject=\mailsubject}
                               {#1}}

% Should use this command wherever the print date is mentioned.
\newcommand{\printdate}{\today}

\subject{451670 - Algorithmen, Wahrscheinlichkeit, Informationen}
\title{Series 4}

\author{Michael Senn \maillink{michael.senn@students.unibe.ch} - 16-126-880}

\date{\printdate}

% Needs to be the last command in the preamble, for one reason or
% another. 
\usepackage{hyperref}


\begin{document}
\maketitle


\setcounter{chapter}{3}
\chapter{Series 4}

\section{Inversions}

Let $\Pi_1 = \{\pi_{1_1}, \ldots, \pi_{1_n}\}$ be a random permutation of $\{1,
\ldots, n\}$. Let $j, k \in \Pi_1, j > k$. Let $\Pi_2$ be equal to $\Pi_1$
except for $j$ and $k$ flipped. Let $X$ be the number of inversions in $\Pi_1$.

Then:
\begin{align*}
	& (j, k)\ \text{permutation of}\ \Pi_1 \\
	\Rightarrow & j > k \\
	\Rightarrow & (j, k)\ \text{not permutation of}\ \Pi_2
\end{align*}

Similarly:
\begin{align*}
	& (j, k)\ \text{not permutation of}\ \Pi_1 \\
	\Rightarrow & j < k \\
	\Rightarrow & (j, k)\ \text{permutation of}\ \Pi_2
\end{align*}

Let:
\[
	I_{j,k} =
	\begin{cases}
		1 & (j, k)\ \text{inversion of}\ \Pi_1 \\
		0 & \text{else}
	\end{cases}
\]

As per above:
\begin{align*}
	& P(I_{j, k} = 1) = \frac{1}{2}\ \forall j, k \\
	\Rightarrow & E(I_{j, k}) = \frac{1}{2}
\end{align*}

And finally:
\begin{align*}
	E[X] & = \sum_{j=1}^n{\sum_{k=j+1}^n{\frac{1}{2}}} \\
	& = \sum_{j=1}^n{\frac{1}{2} (n-j)} \\
	& = \frac{1}{2} \sum_{j=1}^n{n-j} \\
	& = \frac{(n-1) n}{4}
\end{align*}


\section{Random permutations}

Adjusting the algorithm as described does not preserve the property of the
produced permutations following a uniform distribution.

\begin{proof}
	Let $S := \{1, 2, 3\}$. Let $\Pi$ be a random permutation of $S$
	produced by the original algorithm.
	
	In order for $S = \Pi$, the algorithm must, in every iteration $i$,
	produce the random number $j = i$.

	With the original algorithm we have:
	\begin{align*}
		P(S = \Pi) & = P(j = 1 | i = 1) P(j = 2 | i = 2) P(j = 3 | i = 3) \\
		& = \frac{1}{3} * \frac{1}{2} * 1 \\
		& = \frac{1}{6}
	\end{align*}

	Which is the expected result, as there are $3! = 6$ permutations of $S$.

	With the adjusted algorithm we have:
	\begin{align*}
		P(S = \Pi) & = P(j = 1 | i = 1) P(j = 2 | i = 2) P(j = 3 | i = 3) \\
		& = \frac{1}{3} * \frac{1}{3} * \frac{1}{3} \\
		& = \frac{1}{27} \\
		& \neq \frac{1}{6}
	\end{align*}
\end{proof}

\section{Markov-(In)equality}

The Markov inequality fulfills the equality case for a given $m$ for a random
variable $Y$ which takes one of two values $\{0, m\}$.

\begin{proof}
	Let $m \in \mathbb{N}$. Let $Y$ be a random variable which takes one of
	two values $\{0, m\}$.

	We then have:
	\[
		E[Y] = P(Y = 0) * 0 + P(Y = m) m = P(Y = m) m
	\]

	As $Y$ takes one of two values, and $0 < m$:
	\[
		P(Y \geq m) = P(Y = m)
	\]

	As such:
	\[
		\frac{E[Y]}{m} = \frac{P(Y = m) m}{m} = P(Y = m) = P(Y \geq m)
	\]
\end{proof}

\section{Variance}

\subsection{Expected value and variance of discrete uniform distribution}

Let $X \sim Un(\{-n, \ldots, n\})$.

Note that $n + -n = 0\ \forall n \in \mathbb{N}$. Then:
\begin{align*}
	E[X] & = \sum_{i=-n}^n{P(X=i) * i} \\
	& = n * 0 + 0 \\
	& = 0
\end{align*}

Note further that $x^2 = \abs{x}^2\ \forall n \in \mathbb{N}$. Then:
\begin{align*}
	E[X^2] & = \sum_{i=-n}^n{P(X = i) * i^2} \\
	& = 0 + 2 * \sum_{i=1}^n{P(X = i) * i^2} \\
	& = 2 * \sum_{i=1}^n{\frac{i^2}{2n + 1}} \\
	& = \frac{1}{3} n (n + 1)
\end{align*}

And lastly:
\[
	Var(X) = E[X^2] - E[X] = \frac{1}{3} n (n+1)
\]

\subsection{Same minimum and maximum, bigger variance}

Let $Y$ be a random variable, with $P(Y = n) = P(Y = -n) = \frac{1}{2}$. Then:
\begin{align*}
	& E[Y] = 0 \\
	& E[Y^2] = \frac{1}{2} n^2 + \frac{1}{2} n^2 = n^2 \\
	& Var(Y) = n^2 \\
	n \geq 1 \Rightarrow & n^2 > n (n + 1)
\end{align*}

\section{Chebyshev-Inequality}

Let $X \sim Bi(1000, 0.4)$. Hence:
\begin{align*}
	E[X] & = 1000 * 0.4 = 400 \\
	Var(X) & = E[X] * (1-p) = 240
\end{align*}

Then:
\begin{align*}
	P(X < 350 \lor X > 450) & = P(\abs{X - 400} > 50) \\
				& = P(\abs{X - E[X]} > 50) \\
	   & \leq \frac{Var(X)}{50^2} \\
	   & = 0.96
\end{align*}

\end{document}

