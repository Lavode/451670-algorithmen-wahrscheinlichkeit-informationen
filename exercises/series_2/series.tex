\documentclass[a4paper]{scrreprt}

% Uncomment to optimize for double-sided printing.
% \KOMAoptions{twoside}

% Set binding correction manually, if known.
% \KOMAoptions{BCOR=2cm}

% Localization options
\usepackage[english]{babel}
\usepackage[T1]{fontenc}
\usepackage[utf8]{inputenc}

% Enhanced verbatim sections. We're mainly interested in
% \verbatiminput though.
\usepackage{verbatim}

% PDF-compatible landscape mode.
% Makes PDF viewers show the page rotated by 90°.
\usepackage{pdflscape}

% Advanced tables
\usepackage{tabu}
\usepackage{longtable}

% Fancy tablerules
\usepackage{booktabs}

% Graphics
\usepackage{graphicx}

% Current time
\usepackage[useregional=numeric]{datetime2}

% Float barriers.
% Automatically add a FloatBarrier to each \section
\usepackage[section]{placeins}

% Custom header and footer
\usepackage{fancyhdr}

\usepackage{geometry}
\usepackage{layout}

% Math tools
\usepackage{mathtools}
% Math symbols
\usepackage{amsmath,amsfonts,amssymb}

\pagestyle{plain}
% \fancyhf{}
% \lhead{}
% \lfoot{}
% \rfoot{}
% 
% Source code & highlighting
\usepackage{listings}

% Convenience commands
\newcommand{\mailsubject}{451670- Algorithmen, Wahrscheinlichkeit, Informationen - Series 2}
\newcommand{\maillink}[1]{\href{mailto:#1?subject=\mailsubject}
                               {#1}}

% Should use this command wherever the print date is mentioned.
\newcommand{\printdate}{\today}

\subject{451670 - Algorithmen, Wahrscheinlichkeit, Informationen}
\title{Series 2}

\author{Michael Senn \maillink{michael.senn@students.unibe.ch}}

\date{\printdate}

% Needs to be the last command in the preamble, for one reason or
% another. 
\usepackage{hyperref}


\begin{document}
\maketitle


\setcounter{chapter}{1}
\chapter{Series 2}

\section{Throwing dice}

Let $X$, $Y$ be the results of throwing a six-sided dice. Let $Z := X + Y$.

\subsection{One die even}

If $X$ is even then it follows the discrete uniform distribution on $\{2, 4,
6\}$. As such, its expected value is given as $E[X | X \texttt{even}] = \frac{1}{3} * (2
+ 4 + 6) = 4$. Hence:
\[
	E[Z | X \texttt{even}] = E[X | X \texttt{even}] + E[Y] = 4 + 3.5 = 7.5
\]

\subsection{One die odd}

If Y is odd, then $Y \sim Un(\{1, 3, 5\}) \Rightarrow E[Y | \texttt{Y odd}] = 3$.

Hence:
\[
	E[Z | Y \texttt{odd}] = E[Y | Y \texttt{odd}] + E[X] = 3 + 3.5 = 6.5
\]

\subsection{Sum equal to 5}

A sum of $5$ can be achieved via one of four combinations - $\{\{1, 4\}, \{2,
3\}, \{3, 2\}, \{4, 1\}\}$ - all of which are equally likely. Hence:
\[
	E[X | Z = 5] = \frac{1}{4} * (1 + 2 + 3 + 4) = 2.5
\]

\subsection{Sum greater than 6}

\begin{tabular}{|l|r|}
	\hline
	Z & Possible values for X \\
	\hline
	7 & \{1, 2, 3, 4, 5, 6\} \\
	\hline
	8 & \{2, 3, 4, 5, 6\} \\
	\hline
	9 & \{3, 4, 5, 6\} \\
	\hline
	10 & \{4, 5, 6\} \\
	\hline
	11 & \{5, 6\} \\
	\hline
	12 & \{6\} \\
	\hline
\end{tabular}

Hence:
\[
	E[X | Z > 6] = \frac{\sum_{i=1}^6{i^2}}{\sum_{i=1}^6{i}} = \frac{91}{21} = 4.\overline{3}
\]

\section{Jensen squared}

Recall that a function $f(x)$ is twice differentiable, with $f''(x) \geq 0
\forall x$, iff it is convex.

Let $f(x) := x^{2k}, k \in \mathbb{N}$. Then:
\[
	f''(x) = k * (k-1) * x^{k-2} \geq 0
\]

As such, $f(x) = x^k$ is convex for all even $k \geq 2$.

According to Jensen's inequality, for a convex function $f$, and $\lambda_i >
0$ with $\sum_{i=1}^n \lambda_i = 1$, we have:

\[
	f(\sum_{i=1}^n{\lambda_i x_i}) \leq \sum_{i=1}^n{\lambda_i f(x_i)}
\]

Let $\lambda_i := P(X = x_i)$, $f(x) = x^k$. It follows that:
\begin{align*}
	E[X]^k & = (\sum_{i=1}^n{P(X=x_i) * x_i})^k \\
	& = f(\sum_{i=1}^n{\lambda_i * x_i}) \\
	& \leq \sum_{i=1}^n{\lambda_i * f(x_i)} \\
	& = \sum_{i=1}^n{P(X=x_i) * x_i^k} \\
	& = E[X^k]
\end{align*}

\section{Expected value of min/max of IID variables}

\section{Random text generation}

In the following, we assume the cat to produce a constant stream of random
characters from our 32-character alphabet, from which we randomly pick an input
of approriate length.

This does not reflect a realistic scenario, where a cat would generate inputs
of varying length before leaving the keyboard.

This does also not reflect the scenario of a brute-force attack, where one
would start with all inputs of length 1, followed by all inputs of length 2,
and so forth.

This specific difference can be neglected for non-trivially sized alphabets and
input lengths, as the number of possible inputs of length $n$ based on an
alphabet with $x$ characters will quickly converge towards $(x - 1)$ times the
number of possible inputs with length less than $n$:

\[
	lim_{n\to\infty} \frac{x^n}{\sum_{i=1}^{n-1}{x^i}} = x-1
\]

\subsection{Cat attacks 10-char password}

With our 32-char alphabet, there exist a total of $32^{10}$ inputs of length $10$
- of which one is our password. As such, the chance of the cat randomly entering our password is:
\[
	\frac{1}{32^{10}}
\]

\subsection{Cat attacks 128-bit AES key}

With our 32-char alphabet we are able to encode 5 bits per character ($2^5 =
32$). As such, the 128-bit AES key would have to be encoded - with left- or
right-padding - in 26 chars.

As such, the chance of the cat randomly entering the AES key is:
\[
	\frac{1}{32^{26}}
\]

\subsection{Cat attacks 125-char string}

As before, the chance of the cat randomly generating the 125-char string of this exercise's header is:
\[
	\frac{1}{32^{125}}
\]

\subsection{Cat on speed}

In the following we assume the attack to be a targeted brute-force attack,
where the input length is known. As this prevents any input sequence from being
tried multiple times, half the possible inputs will have to be tried on
average.

If the inputs were generated randomly and repeating inputs were allowed, then
the average number of inputs one has to generate to find one equal to the
target is given as the minimum value of $x$ such that $(1-p)^x \leq 0.5$, with
$p$ being the probability of one input being equal to the target.

If inputs were generated with a frequency of $f := 40 * 10^{18}\ \texttt{Hz}$,
then it would, on average, take:

For a targeted attack on the 10-char password:
\[
	\frac{32^{10}}{2} * f^{-1} = 1.407 * 10^{-5}\ \texttt{s}
\]


For a targeted attack on the 128-bit AES key:
\[
	\frac{32^{26}}{2} * f^{-1} = 1.701 * 10^{19}\ \texttt{s, approx 120 times the age of our sun}
\]


For a targeted attack on the 125-char heading:
\[
	\frac{32^{125}}{2} * f^{-1} = 1.74 * 10^{168}\ \texttt{s, approx } 4 * 10^{150}\ \texttt{times the age of the universe}
\]


\end{document}

