\documentclass[a4paper]{scrreprt}

% Uncomment to optimize for double-sided printing.
% \KOMAoptions{twoside}

% Set binding correction manually, if known.
% \KOMAoptions{BCOR=2cm}

% Localization options
\usepackage[english]{babel}
\usepackage[T1]{fontenc}
\usepackage[utf8]{inputenc}

% Enhanced verbatim sections. We're mainly interested in
% \verbatiminput though.
\usepackage{verbatim}

% PDF-compatible landscape mode.
% Makes PDF viewers show the page rotated by 90°.
\usepackage{pdflscape}

% Advanced tables
\usepackage{tabu}
\usepackage{longtable}

% Fancy tablerules
\usepackage{booktabs}

% Graphics
\usepackage{graphicx}

% Current time
\usepackage[useregional=numeric]{datetime2}

% Float barriers.
% Automatically add a FloatBarrier to each \section
\usepackage[section]{placeins}

% Custom header and footer
\usepackage{fancyhdr}

\usepackage{geometry}
\usepackage{layout}

% Math tools
\usepackage{mathtools}
% Math symbols
\usepackage{amsmath,amsfonts,amssymb}
\usepackage{amsthm}

\DeclarePairedDelimiter\abs{\lvert}{\rvert}

\pagestyle{plain}
% \fancyhf{}
% \lhead{}
% \lfoot{}
% \rfoot{}
% 
% Source code & highlighting
\usepackage{listings}

% Convenience commands
\newcommand{\mailsubject}{451670- Algorithmen, Wahrscheinlichkeit, Informationen - Series 7}
\newcommand{\maillink}[1]{\href{mailto:#1?subject=\mailsubject}
                               {#1}}

% Should use this command wherever the print date is mentioned.
\newcommand{\printdate}{\today}

\subject{451670 - Algorithmen, Wahrscheinlichkeit, Informationen}
\title{Series 8}

\author{Michael Senn \maillink{michael.senn@students.unibe.ch} - 16-126-880}

\date{\printdate}

% Needs to be the last command in the preamble, for one reason or
% another. 
\usepackage{hyperref}


\begin{document}
\maketitle


\setcounter{chapter}{7}
\chapter{Series 8}

\section{Probability of given vertex selected by algorithm}

The algorithm $S$ will select a given vertex $v \in V$ exactly if in its first
step it selects an edge $e \in E$ of which one component is equal to $v$, and
in its second step choses $v$ rather than the second component of the selected
edge.

As there are a total of $m$ edges, of which $deg(v)$ have $v$ as one of their 
components, the probability is given as:

\begin{align*}
	P[S = v] & = \frac{deg(v)}{m} \cdot \frac{1}{2} \\
		 & = \frac{deg(v)}{2m}
\end{align*}

\section{Expected value of size estimator}

Let $l(v) := \frac{2m}{deg(v)}$. By the definition and properties of the
expected value, $E[l(S)]$ is given as:

\begin{align*}
	E[l(S)] & = \sum_{s \in S}{P(S = s) \cdot l(s)} \\
		& = \sum_{s \in S}{\frac{deg(s)}{2m} \cdot \frac{2m}{deg(s)}} \\
		& = \sum_{s \in S}{1} \\
		& = n
\end{align*}

Hence, $l(v)$ is an unbiased estimator for $n$, the size of the graph.

\end{document}

